\documentclass[12pt]{article}
\usepackage{graphicx}
\usepackage{tikz}
\def\cm{\tikz\fill[scale=0.4](0,.35) -- (.25,0) -- (1,.7) -- (.25,.15) -- cycle;}
\title{\textbf{Hanabi PRD}}
\author{Mels Kolozyan\\
		Xudong Li\\
		Xufeng Lin\\
		Evan Semenoff\\
		Tiandu Xie\\
		}
\date{}
\begin{document}

\maketitle
\newpage
\tableofcontents

\newpage

\begin{center}

\large{Executive Summary:}
\end{center}

Hanabi is a cooperative, solitaire-like game based around logical inference.
Players give hints to help each other reason about hidden information, and use a
combination of deductions, assumptions, and implications in an attempt to complete
the game together.

This document outlines the requirements for the implementation of a game client
by which players can enjoy the game of Hanabi over the internet.  The goal is to make
an efficient, aesthetic, and well-featured piece of software that captures the
original spirit of the card game it was based on.


\newpage


\part{Introduction}

\section{Purpose}
\subsection{What is the Purpose of this Document?}

This document describes the full requirements necessary for a client-server
based implementation of the card game Hanabi.  It follows (as much as
applicable) the standards set out by the IEEE to address the concerns of what
the client will be capable of, what is required by it, and how those concerns
will be addressed.

In addition to the requirements of the software from a functional perspective,
it also outlines the verification requirements that the software will be subject
to later in development.

\section{Scope}
\subsection{Full Software Description}

The general scope of the software is split into two distinct modes:

\paragraph{Client Mode:} This mode is the main 'game' mode of the software. In general,
it allows a user to play Hanabi through a GUI based client.  
In this mode a player will be able to both create, join, and play games
of Hanabi.  Additionally, a basic tutorial will available to explain the rules of the game.
It's important to note that this mode
is not responsible for the implementation of the game, it is responsible strictly for
the presentation of the current game state, allowing the user to interact with it.

\paragraph{AI Mode:} This mode will allow a user to connect a computer player
to an existing game. It is to be implemented to allow execution directly from the command
line. After being sent to a game, the AI will play until the game is completed.

\subsection{Software Features and Limitations}

\subsubsection{Client Features}

\begin{description}

\item [a)]
The player will be given a simple, informative GUI that is both intuitive and
clutter-free.  The UI will focus heavily on helping the player keep track
of the game state by allowing them to mark inferences on their cards. It will also
guide the player by highlighting legal moves during gameplay, with the goal
of making the game more accessible.

\item [b)]
In order to aid in such accessibility, the game will also feature some basic
animations to help the player keep track of the game state.  Two types of animations
that will be important will be:

\begin{description}
\item[i.] Animations that help the user understand what they're interacting with in
the UI by doing things like highlighting buttons that are being hovered over, 
shifting objects when the cursor is placed over-top, or animating buttons when
they're clicked on.

\item[ii.] Animations that help the user understand the state of the game.  Since
information conveyance is so important in Hanabi, these will have to be
clear.

\end{description}
\item [c)]
Simple sounds like clicking and shuffling noises that give a player
clear audio feedback when an action has been performed.
\end{description}

\subsubsection{Client Limitations}
\begin{description}
\item[a)]
Although the game will have a GUI and basic animations, the general art
design of the game will be kept minimalistic to fit within the time
constraints of the project.
\item[b)]
The in-game client options should be kept as minimal as possible, although
the intention is to add a small set of utility functions to the player, too many
would not only obfuscate the clarity of the interface but also add unnecessary 
complexity to the program.
\end{description}

\subsubsection{AI Features}
\begin{description}
\item[a)]The AI player should be able to interact meaningfully with any game it is told to
connect to and play a game from start to finish without causing a timeout.

\item[b)]The AI player should perform objectively better than an AI which simply generates
a random move and plays it if it's legal.

\end{description}

\subsubsection{AI Limitations}
\begin{description}
\item[a)]The AI player will have relatively low resource usage. The client should will
run any pre-built computer produced after 2010.

\item[b)]The AI player will not play a perfectly optimal move at all times, it
doesn't need to perfectly solve the entire game of Hanabi.

\end{description}


\section{Definitions}
\subsection{Vocabulary}
\begin{itemize}
\item[Action] A specific movement depended on different actors
\item[Actor] A role in the whole system, including clients and the server
\item[Bug] Error or failure of the program 
\item[Fuse] A type of game tokens to indicate remaining mistakes allowed
\item[Heuristic] Evaluation system providing ideas how the artificial intelligence will choose a move
\item[Library] A collection of code invoked to construct the program
\item[Packet] A data packet,the smallest unit of internet data transferred between servers and clients
\item[Protocol] Rules and regulations for internet networking
\item[Server] Responds the request from clients, it sets games up and run all processes for programs interacting with the clients
\end{itemize}
\subsection{Acronyms}
\begin{itemize}
\item[AFK] Away from keyboard
\item[BOK] Back on keyboard
\item[FPS] Frames per second
\item[GM] Game master
\item[GUI] Graphical user interface
\item[IEEE] Institute of electrical and electronics engineers
\item[IP] Internet protocol
\item[J2SE] Java 2 standard edition
\item[JVM] Java virtual machine
\item[TCP] Transmission control protocol
\item[UI] User interface
\end{itemize}

\subsection{System Overview}
The system, in full, should comprise of two main functional components.
\begin{description}
\item[The Client]
The component that allows an end user to play a game of Hanabi with
a group of other users over a network connection.
\item[The AI]
A program that can be run which mimics the behaviour of a player
after connecting with a server.
\end{description}

\subsection{Documentation Overview}
This document is organized to outline the basic requirements of the system by
starting off with generalized, abstract requirements, and refining them to
down to their more specific, and concrete parts.

Section 2.0 is concerned with the overall description of the system to be 
delivered, and will provide an outline of the actors and their use cases, the constraints
around which the system will be designed, and finally the set of hardware and 
software requirements for the end user.

Section 3.0 addresses in more detail, the use cases outlined
in section 2.0, and expands these use cases
into interfaces, functional, performance, and
maintainability requirements.

Finally, section 4.0 addresses how the requirements of section 3.0 will be
verified in a reliable, quantifiable way.

\section{References}
\subsection{Textbooks and Papers}
\begin{itemize}
\item David C. Kung; Object-Oriented Software Engineering: An Agile Methodology; McGraw-Hill, 2013

\item Olivieri, J. (2009). Quantifying Software Reliability and Readiness. 
        Retrieved from MITRE: http://www.asq509.org/ht/a/GetDocumentAction/i/46491
        
\item Steve McConnell; Code Complete 2; Microsoft Press, 2004
\end{itemize}

\subsection{Technical Documentation}
\begin{itemize}
\item Dutchyn, C. (2019, January 22). Intermediate Software Engineering Assignment 1–Requirements.

\item Systems and software engineering — Life cycle processes — Requirements engineering. 
        (2011, December 01). International Standard. Switzerland : ISO/IEC/IEEE.
        
\item IEEE Recommended Practice for Software Requirement Specifications. (1998, October 20). 
        IEEE Computer Society. Software Engineering Standards Committee.
\end{itemize}

\newpage
\part{Overall Description}
\setcounter{section}{0}
\section{Product Functions}

This part outlines the overall requirements of the program, it covers the
main users (actors) and their interactions with the software.  It addresses
the use cases for said actors and outlines a generalized data flow for the
software.  Finally, it will define the constraints model that these requirements
will fall under.

\subsection{Actors}


There are three actors in this program:

\begin{enumerate}
\item
A game master (GM) is the creator of the table and is responsible for specifying 
the number of players (including themselves) that will be playing. The action of 
joining a table is not possible for a GM, because as soon as they request the 
creation of a new table they automatically join this new table. Other than the
pre-game setup a GM's capabilities and possible actions are identical to those
of a regular player.

\item
A regular player's available pre-game action is joining a game, which includes
connecting to an already created server through a game-ID. 

\item
An AI bot is invoked by a human and given unique identifying information along
with a game-id, which enable it to join the desired server. Upon joining, the
AI is expected to play until the game is terminated. 
\end{enumerate}


\subsection{Use Cases and Operating Concepts}

With the actors identified, this section will focus on outlining their
basic interactions with the software by specifying use cases for each
actor, and then give a brief overview of what these actions entail.

\begin{enumerate}
\item GM Actions:
\begin{itemize}
\item Create table
\item Play a card
\item Give a hint
\item Discard a card
\item View discarded cards
\item Create personal notes for each card
\item Start the AI (AFK)
\item Stop the AI (BOK)
\item Exit table
\end{itemize}
\item Regular Player Actions:
\begin{itemize}
\item Join table (player)
\item Play a card
\item Give a hint
\item Discard a card
\item View discarded cards
\item Create personal notes for each card
\item Start the AI (AFK)
\item Stop the AI (BOK)
\item Exit table
\end{itemize}  
\item AI:
\begin{itemize}
\item Join table (AI)
\item Play a card
\item Give a hint
\item Discard a card
\item View discarded cards
\end{itemize}
\end{enumerate}

\subsubsection{Action Definitions}

\begin{itemize}

\item [\textbf{Create Table}]
	When the user is performing this action, they are adopting the 'Game Master'
	roll. 

	This action allows a player to create a table.  They will do so by providing
	some basic information to the client such as:
	\begin{itemize}
	\item[a)] Number of players
	\item[b)] Time limit for making a move
	\item[c)] Server name
	\end{itemize}
	

\item[\textbf{Join Table}]

This is the basic action which allows either a human player or a computer player
to join a table.  Although they're essentially the same thing from the point of view
of the server, they have slightly different usages.
	\begin{itemize}
    \item[a)](player) A player can join a server by specifying the game-ID. After this 
    they are able to do all of the list in-game actions.
    \item[b)](AI) A human invokes the AI with a game-ID and identifying 
    token to join a server via the command line.
	\end{itemize}

\item[\textbf{Play Card}]
    This is the basic action of playing a card. After selecting a card to play, 
    the card is then placed on the table face-up.  It is then moved to either
    the discard pile or the play area depending on the validity of the move.
    If the move was invalid, 
    the card is moved to the discard pile and the number of fuses decreases by
    one. This action consumes a turn, all players are informed of the play and
    gameplay resumes with the next player.

\item[\textbf{Give Hint}]
	This is the action which allows one player to communicate information with another player.
	The hint may be about 
    either the colour or the number of the card(s), but not both. 
    This action consumes a turn, all players are informed of the play and
    gameplay resumes with the next player.

\item[\textbf{Discard}]
	This action allows a player to discard a card from their own hand if the
	team does not have the maximum amount of information tokens.  The player
	will select a card and then select the discard option. The selected card
	will be added to the discarded cards pile, and one information token will be
	restored. This action consumes a turn, all players are informed of the play and
    gameplay resumes with the next player.
	
    
\item[\textbf{View Discards}]
    At any point during the game, it is possible to view all the cards
    currently in the discard pile. Selecting this option during
    gameplay will bring up a graphical list of cards for the player to examine.

\item[\textbf{Create Notes}]
    This function will allow players to keep track of their own inferences
    about their cards by giving them an interface which allows them to record 
    either an assumed number or colour.

\item[\textbf{AFK}]
    At any point during the game a player can activate the AFK (Away From 
    Keyboard) option, which invokes an AI bot that plays for the player while
    they are gone. 

\item[\textbf{BOK}]
    If the player has switched to AFK mode and an AI is playing for them, this
    function returns the control of their hand to them.

\item[\textbf{Exit Table}]
    After selecting this action the connection to the server is severed and
    the player is returned to the main screen.
\end{itemize}

The game is terminated when all of the in-game fuses run out, there are no more
cards to draw (each player is allowed one more turn after this happens) or one or 
more of the players disconnect from the server.

\subsection{Use Case Diagram}

\includegraphics[height=12cm]{f1.png}
\subsection{Data Flow Diagram}
\includegraphics[height=12cm]{f2.png}
\newline
This diagram outlines a round of play, and shows how the control is switched from player
to player.

\section{User Characteristics}

It will be assumed that most users will have basic computer usage skills such as:
\begin{itemize}
\item General mouse usage (can click, drag-and-drop, etc.)
\item General keyboard usage
\item Interface knowledge (understands that things that are greyed out are unavailable)
\end{itemize}
It will also be assumed that users who wish to use the AI functionality of the software
(sending an AI to play an entire game) will have have the ability to use a command line.


\section{Constraints}

This section outlines the basic hardware and software constraints that
will be required of the end user in order to run the software.

\subsection{Platform}

The software will be designed to run on the JVM and should run on any machine
with an installation of the J2SE (Java Platform, Standard Edition). This should
allow for the software to run on most standard, modern operating systems such
as MacOS, Windows, and Linux.

\subsection{Network Protocols}

The software will use a standard network protocol to communicate with the
server.  This means that support for TCP will be required.

A two-way network connection is required to allow clients to ask requests 
and receive messages from the server. The networking protocols used for 
communication between clients and the server of the system should guarantee 
data packets being sent and received.

\subsection{Timeframe}

The timeframe for the project is three months.

\section{Assumptions and Dependencies}

This section covers the general assumptions that are made during the requirements
phase, and the dependencies that will most likely arise during the development,
and why they will be included in the project.

\subsection{Libraries}

\begin{enumerate}
\item \textbf{Java Swing}:
This library will support general GUI design of the software. It will
be necessary for basic dialogue boxes, error reporting, and window drawing.
\item \textbf{JavaFX}:
This library will be used to draw the visual portion of the game
\item \textbf{EJML}:
This library will be used by the AI for matrix calculations, as well
as keeping track of various local game-state variables during play.
\item \textbf{Java.net}:
This will be the library responsible for addressing the general
networking concerns of the software.
\item \textbf{AspectJ}:
This extension of Java will aid in developing the software from an
aspect-oriented approach.
\end{enumerate}

\newpage
\part{Specific Requirements}
\setcounter{section}{0}
\section{External Interfaces}

After launching the application, the user will be presented with a title screen 
(fig. 3, Appendix A). The options given on this screen will allow users to create a new 
game table, join an existing game, start a tutorial, change 
some basic game settings, and lastly, exit the game.

\subsection{User Interface}
\begin{itemize}



\item[a)] Title Screen Interface:



If the user chooses to create a new game table, a dialogue box for creating games
will be displayed (fig. 4, Appendix A). Users need to enter name of the game table and 
the maximum number of players that will be allowed in this game. Finally,
a create game button and a cancel button will be provided.

Next, if the player selects the 'join an existing game' option they will again
be shown a dialogue box.  This dialogue box will prompt the user for the name (or
address of the game they wish to join) and will provide the user a cancel
button that closes the dialogue box.

If the user selects 'options', then the game will display a series of local
options that can be modified by the player.  This will allow them to slightly
customize their client.

If the user selects 'tutorial', the game will show a short animation of a few
rounds of play, explaining the basic rules of the game.

Lastly, if the user selects 'exit game', the client will close gracefully.

\item[b)] Game Table Interface:

After successfully creating a game or joining an existing game, the basic game
interface will be displayed (fig. 5, Appendix A). The table will be separated into two
main parts. The left part of the screen will be the main game table, it is the
primary area of play (akin to a real table). Each (other) player will have an 
area on the screen where they're identified, and where the player (client) will
be able to see their cards. 

Cards in the player's (client) hand will be placed directly at the bottom of the
screen. At the top right corner of each card, there will be 
two small tag areas (fig. 6, Appendix A) which allow the player to mark some simple deductions
on the back of the card. In the middle area, all cards that have been successfully 
played will be put in order and distinguished by colour. 


The right side of the screen will be utilized as a 'game status' area.
From top to bottom, all discards can be selected to view, 
time remaining will indicate time left in current turn, the number of cards left 
will also be shown. In addition, the number of hint tokens and fuses
left on the table will also be displayed. 

\end{itemize}

\subsection{Hardware Interface}

Since the application does not require any application specific hardware, this
section will be brief. The basic hardware requirements are addressed by a keyboard
and mouse.  The network interfacing can be done by any standard, modern, network
interface card.

\subsection{Communications Interface}

Communication between different interfaces would be handled by the underlying 
operating system, and the communication between clients and server would use 
a common network protocol.

\section{Functions}
\subsection{Validity Checking Inputs}


The system shall check the inputs from the user. At any point in the control flow
of the software, where user input is required, validity checks will be done.


\subsection{Sequence of Operations}

This section serves to outline the basic pre and postconditions for
each of the major operations.


\begin{description}
\item
\begin{description}
	\item \textbf{Create Game:}
	\item Preconditions:
		\begin{itemize}
		\item The user is not in a game
		\end{itemize}	
	\item Postconditions:
		\begin{itemize}
		\item A new game is created, and the user is automatically joined
		\end{itemize}
\end{description}

\item
\begin{description}
	\item \textbf{Join Game:}
	\item Preconditions:
		\begin{itemize}
		\item The user is not in a game
		\end{itemize}	
	\item Postconditions:
		\begin{itemize}
		\item The user is entered into a game
		\end{itemize}
\end{description}

\item
\begin{description}
	\item \textbf{Exit:}
	\item Preconditions:
		\begin{itemize}
		\item None
		\end{itemize}	
	\item Postconditions:
		\begin{itemize}
		\item The game client is closed
		\end{itemize}
\end{description}

\item
\begin{description}
	\item \textbf{Discard:}
	\item Preconditions:
		\begin{itemize}
		\item The user is in a game
		\item The user has selected a card
		\item Less than eight information tokens are available
		\end{itemize}	
	\item Postconditions:
		\begin{itemize}
		\item The selected card is added to the discard pile
		\item A new card is added to the players hand
		\item One information token is replenished
		\end{itemize}
\end{description}

\item
\begin{description}
	\item \textbf{Play Card:}
	\item Preconditions:
		\begin{itemize}
		\item The user in in a game
		\item The user has selected a card
		\end{itemize}	
	\item Postconditions:
		\begin{itemize}
		\item The selected card is played
		\item A new card is added to the players hand
		\end{itemize}
\end{description}

\item
\begin{description}
	\item \textbf{Give Hint:}
	\item Preconditions:
		\begin{itemize}
		\item The user has selected a card to give a hint about
		\item The user has selected the type of hint to give (colour or number)
		\item At least one information token is available
		\end{itemize}	
	\item Postconditions:
		\begin{itemize}
		\item An information token is used
		\item The selected player is given the hint information
		\end{itemize}
\end{description}

\item
\begin{description}
	\item \textbf{AFK:}
	\item Preconditions:
		\begin{itemize}
		\item The user is in a game
		\end{itemize}	
	\item Postconditions:
		\begin{itemize}
		\item An AI player takes over for the player
		\item The user is put into the AFK state
		\end{itemize}
\end{description}

\item
\begin{description}
	\item \textbf{BOK:}
	\item Preconditions:
		\begin{itemize}
		\item The user is in the AFK state
		\end{itemize}	
	\item Postconditions:
		\begin{itemize}
		\item The AI relinquishes control to the user
		\item The user is taken out of the AFK state
		\end{itemize}
\end{description}


\item
\begin{description}
	\item \textbf{Start AI:}
	\item Preconditions:
		\begin{itemize}
		\item A game exists for the AI to join
		\end{itemize}	
	\item Postconditions:
		\begin{itemize}
		\item The AI plays a game from start to finish
		\end{itemize}
\end{description}

\item
\begin{description}
	\item \textbf{View Discards:}
	\item Preconditions:
		\begin{itemize}
		\item The user is in a game
		\item At least one card has been discarded
		\end{itemize}	
	\item Postconditions:
		\begin{itemize}
		\item The user is presented with a list of discarded cards
		\end{itemize}
\end{description}


\item
\begin{description}
	\item \textbf{Take Notes:}
	\item Preconditions:
		\begin{itemize}
		\item The user is in a game
		\item The user has selected a card
		\end{itemize}	
	\item Postconditions:
		\begin{itemize}
		\item A card in the users hand is marked with some information
		\end{itemize}
\end{description}

\end{description}


\subsection{Responses to Abnormal Situations}

This section outlines the typical responses to abnormal working behaviour
in the program. 

\subsubsection{Network Communication Errors}
When any network communication fails, the client will attempt to resend a message
to the server.  If, after a certain amount of time no response is received, the
client will again attempt to send a message to the server.  If after these two
attempts it receives to acknowledgement, then the client will consider the connection
broken and exit the game (returning to the main screen).

\subsubsection{Error Handling and Recovery}

We will consider three primary types of errors:

\begin{itemize}
\item Insignificant Errors: These errors are known bugs that should not affect the
functionality of the game in a major way. These will include things like:
	\begin{itemize}
	\item Minor graphical issues
	\item Minor display issues
	\item Typos or grammatical errors
	\end{itemize}

\item Minor Errors: These errors are known bugs that do not affect the functionality
of the software in a major way, but potentially make it more inconvenient to use.
This will include bugs like:
	\begin{itemize}
	\item Mislabelled icons
	\item Graphical issues which obfuscate information
	\end{itemize}
	
\item Major Errors: These errors are both known, and unknown bugs which will either
cause the software to stop working entirely, or stop working as intended. If a known
bug in this category is encountered during runtime, the software will report an error
code and attempt to exit gracefully. This category includes errors such as:
	\begin{itemize}
	\item Any standard Java runtime error such as Array Index Out-of-bounds
	\item Any complete networking failure
	\item Graphical issues which cause the program to be impossible to use
	\end{itemize}
\end{itemize}

\section{Performance Requirements}
\subsection{Number of Users}
The client will support a direct connection between two users.  A server, and
an end user.  The server itself will be responsible for handling multiple users,
but from the perspective of any individual client running on a machine, there
are only ever two users connected to it at any given time.

\subsection{Information Handling}
The client will be responsible for communicating with the server, and receiving
communications from the server.  The parsing of information from the server
format will need to be fast enough to create a smooth experience for the end
user.  Although "smooth experience" is hard to quantify, it should still
be verifiable in the testing phase.

When the client is running in AI mode, it will need to be able to calculate
its best move and send a properly formatted packet in the given time-out 
timeframe.  This means that the basic functionality of the AI must be reasonably
efficient given its limited time resources.

\section{Adaptability Requirements}
Any change in the requirements of the project will start a process in which
the change is incorporated by editing all documentation starting with the
requirements, all the way to the current project state.  Changes should be
subject to the same rigour as any feature that was introduced in the initial
staging of the project, and be brought into the working feature set with the
same process as any other feature.


\newpage
\part{Verification Requirements} 
This part will outline the various verification requirements that will be imposed on
the functional parts of the software.  It will describe quantifiable verification
methods for all operations that have failure states, as well as methods to
gauge progress in the harder to quantify aspects of the program such as GUI, usability,
maintainability, and adaptability.
\setcounter{section}{1}



\subsection{User Interface}
The verification of the user interface would be done by walking through every part
of the game. For each different screen or dialogue box, every button or selection
option that is provided will be tested in several conditions. This will allow verification
that all graphical components are working correctly

\subsection{Hardware Interface}
A hardware that just meet the minimal requirements will be provided to run the 
game application. If the application runs well on this hardware, it will be 
considered as appropriate to run at any hardware that would have better factors.
+
\subsection{Communications Interface}
The application will be tested on different operating system that all able to
use common network protocol such as TCP/IP.




\section{Functions}  
\subsection{Validity Checking Inputs}
The verification of correctness for validity checking will be done by attempting
a range of (known) invalid inputs for every functional component that takes user
input.  The test cases for a functional component will be designed around trying
to capture as many edge-cases as possible.  The function will be considered
verified if it ignores all invalid input in the testing suite. The general set
of cases to be considered will be highlighted in parallel to the sequence of
operations section.

\subsection{Sequences of Operations}

This section will highlight the type of input validation that will be done
for various operations.  Note that any function which does not have any invalid
input cases (like exiting the game) are not included.

\begin{description}
\item
\begin{description}
	\item \textbf{Create Game:}
	\item Invalid Inputs:
		\begin{itemize}
		\item Connect to a non-existent server
		\item Overflow the text input field
		\item Make a game with too many players (more than 5)
		\item Make a game with too few players (1)
		\end{itemize}	
\end{description}

\item
\begin{description}
	\item \textbf{Join Game:}
	\item Invalid Inputs:
		\begin{itemize}
		\item Connect to a non-existent server
		\item Connect to a non-existent game
		\item Overflow the text input field
		\item Join a game which is full
		\end{itemize}	
\end{description}	


\item
\begin{description}
	\item \textbf{Discard:}
	\item Invalid Inputs:
		\begin{itemize}
		\item Attempt to discard a card when the token pile is full
		\item Attempt to discard a card which the player does not hold
		\end{itemize}	
\end{description}	

\item
\begin{description}
	\item \textbf{Play Card:}
	\item Invalid Inputs:
		\begin{itemize}
		\item Attempt to play a card which the play does not hold
		\end{itemize}	
\end{description}	

\item
\begin{description}
	\item \textbf{Give Hint:}
	\item Invalid Inputs:
		\begin{itemize}
		\item Attempt to give a hint when no information tokens are available
		\item Attempt to give an invalid hint (both colour and number)
		\end{itemize}	
\end{description}	
\end{description}

\subsection{Responses to Abnormal Situations}

\subsubsection{Overflow and Underflow}
To verify that overflow and underflow type situations are handled properly,
any function which deals with numerical values will be put through a testing
suite consisting of various overflow/underflow situations.  If a function
behaves robustly in all cases, it will be considered verified with respect to this section.
\subsubsection{Network Communication Errors}
Verification of correct behaviour during network communication errors will be
done by confirming attempts to reconnect (resend) network information within
a given time frame. Once the software is able to detect an error and respond
accordingly within a specified real-time timeframe, it will be considered
verified.
\subsubsection{Error Handling and Recovery}
Finally, confirming that error handling and recovery happen in an appropriate
manner can only be done on a case-by-case basis.  When an error is discovered,
it will initially be 'patched' in an attempt to handle and isolate the error,
this will be categorized (severity) and outlined (nature).  The bug will now
be considered "known", and appropriate recovery code will be called when this bug
is triggered.  The appropriate recovery response will be relative to the 
severity of the error.  Confirming that this system is correct will be done by
creating test cases consisting of "fake bugs", that trigger the desired response.


\section{Performance Requirements}

This section aims to outline the verification process for the performance aspect
of the software.  

\subsection{Information Handling}

To verify the performance aspect of the networking function, a series of tests
that aim to quantify the network latency will be performed.  The goal will be
to get the client behaving in a way that most users would agree is "smooth".
Although this is hard to quantify directly, a series of user-experience based
questions should give adequate information on if we're heading in the correct
direction. Some examples of questions that could aid in quantifying the
performance of the game are:
\begin{enumerate}
\item Did the game feel more, or less responsive?
\item Were you frustrated with the loading times?
\item Where can the experience made to be more enjoyable (from the perspective of performance)
\item Did you notice any considerable slowdown at any point?
\end{enumerate}
After every build that addresses the issues of performance, the team will 
discuss how the new build felt, and if the alterations to the client side
performance were an enhancement or a detriment to the enjoyability of using
the application.

Another tool that could be looked at to help determine if performance 
requirements have been met are examining the FPS,the network
response time, and the profile of the software while running, but
these are only tools to help guide the process, as verification will be
considered complete when a unanimous decision is reached within the team.


When running in AI mode, it will suffice to generate tests that will quantify
response time, and use this as a metric to verify that the AI can, within some
level of certainty, respond to the server with a valid move in a given period
of time.

\section{Maintainability Verification}
A maintainability index will be calculated using both lines-of-code and
number-of-modules as a pair of quantifiable metrics which will attempt to predict
the amount of effort that will be required to maintain the program. The goal 
will not be to explicitly use this index as a measure of success but as a rough
heuristic to help guide development.

The application of this metric should help to maximize the useful life, 
efficiency, and safety of the software, while leaving some space for future 
improvements. Keeping the project simple should facilitate easier bug fixes
in shorter amounts of time.

\section{Reliability Verification}
A simple reliability index will be used to quantify the readiness of the system.
First, a set of quantifiable metrics will be used to guide the process of 
reliability engineering: failure rate, known remaining bugs, and testing 
coverage.  Secondly, the individual issues which these categories address
will be weighted based on the severity of the bug, ranging from serious (the
bug causes an unexpected fatal error) to trivial (the bug causes a small
graphical error for a moment).

This scheme aims to guide the design and testing of a reliable system and giving
a valuable metric to measure progress and quality with.  However, like the
maintainability index, the goal is not to maximize this value, but to
use it as an aid to help guide the engineering process.

%\newpage
\part{Appendices}
\section{Appendix A: Diagrams and Figures}


\includegraphics[height=12cm]{f1.png}
\newline
Figure 1, Use-Case Diagram

\includegraphics[height=12cm]{f2.png}
\newline
Figure 2, Data Flow Diagram

\includegraphics[height=12cm]{3f1.png}
\newline
Figure 3, Title Screen

\includegraphics[height=12cm]{3f2.png}
\newline
Figure 4, Create Game Dialogue

\includegraphics[height=10cm]{3f3.png}
\newline
Figure 5, Game Table

\includegraphics[height=5cm]{3f4.png}
\newline
Figure 6, Player Hand

\section{Appendix B: Comparison between this Document and the ISO standards}
This appendix will outline the differences between the ISO standardization for the
requirements documentation, and this documentation.  The sections that were not included
from the ISO were taken out because they didn't apply to the software as a whole, or
because they were too specific for business application.  Some extra sections were
added that are not part of the SRS standard, these sections were added to help aid
in the verification stage of the project, giving the team a general idea for
verification requirements.

\newpage
\begin{tabular}{|l|c|c|}  
    \cline{2-3}
    \multicolumn{1}{c|}{} & SRS Standard & Hanabi Req. \\
    \hline

    Part I Introduction & \cm & \cm \\
    \hline
    --1. Purpose & \cm & \cm \\
    \hline
    --2. Scope & \cm & \cm \\
    \hline
    --3. Definitions & \cm & \cm \\
    \hline
    --4. References & \cm & \cm \\
    \hline
    --5. Overview & \cm & \cm \\
    \hline

    Part II Overall Description & \cm & \cm \\
    \hline
    --1. Product Perspective & \cm &  \\
    \hline
    --2. Product Functions & \cm & \cm \\
    \hline
    --3. User Characteristics & \cm & \cm \\
    \hline
    --4. Constraints & \cm & \cm \\
    \hline
    --5. Assumptions and Dependencies & \cm & \cm \\
    \hline
    --6. Apportioning of Requirements & \cm & \\
    \hline

    Part III Specific Requirements & \cm & \cm \\
    \hline
    --1. External Interfaces & \cm & \cm \\
    \hline
    --2. Functions & \cm & \cm \\
    \hline
    --3. Performance Requirements & \cm & \cm \\
    \hline
    --4. Logical Database Requirements & \cm & \cm \\
    \hline
    --5. Design Constraints & \cm & \cm \\
    \hline
    --6. Software System Attributes & \cm & \cm \\
    \hline
    ---6.1. Reliability & \cm & \cm \\
    \hline
    ---6.2. Availability & \cm &  \\
    \hline
    ---6.3. Security & \cm &  \\
    \hline
    ---6.4. Maintainability & \cm & \cm \\
    \hline
    ---6.5. Portability & \cm &  \\
    \hline
    ---6.6. Adaptability &  & \cm \\
    \hline
    --7. Organizing the Specific Requirements & \cm &  \\
    \hline
    --8. Additional Comments & \cm &  \\
    \hline

    Part IV Verification Requirements &  & \cm \\
    \hline
    --1. External Interfaces &  & \cm \\
    \hline
    --2. Functions & & \cm \\
    \hline
    --3. Performance Requirements &  & \cm \\
    \hline

    Part V Supporting Information & \cm & \cm \\
    \hline
    --1. Table of Contents & \cm & \cm \\
    \hline
    --2. Appendices & \cm & \cm \\
    \hline

\end{tabular}



\end{document}